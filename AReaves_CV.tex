%%%%%%%%%%%%%%%%%%%%%%%%%%%%%%%%%%%%%%%%%
% Medium Length Professional CV
% LaTeX Template
% Version 2.0 (8/5/13)
%
% This template has been downloaded from:
% http://www.LaTeXTemplates.com
%
% Original author:
% Trey Hunner (http://www.treyhunner.com/)
%
% Important note:
% This template requires the resume.cls file to be in the same directory as the
% .tex file. The resume.cls file provides the resume style used for structuring the
% document.
%
%%%%%%%%%%%%%%%%%%%%%%%%%%%%%%%%%%%%%%%%%

%----------------------------------------------------------------------------------------
%	PACKAGES AND OTHER DOCUMENT CONFIGURATIONS
%----------------------------------------------------------------------------------------

\documentclass{resume} % Use the custom resume.cls style
\usepackage{hyperref}
\usepackage[left=0.75in,top=0.6in,right=0.75in,bottom=0.6in]{geometry} % Document margins
\usepackage{enumitem}%http://ctan.org/pkg/enumitem
\usepackage{amsmath}
\usepackage{fontawesome}
\newcommand{\tab}[1]{\hspace{.2667\textwidth}\rlap{#1}}
\newcommand{\itab}[1]{\hspace{0em}\rlap{#1}}
% Your name
%\usepackage{color}
%\hypersetup{colorlinks=true, urlcolor = blue, linkcolor = black}


\name{Alexander Michael Reaves} 

% Your address
\address{\faEnvelope
\href{mailto:amreaves120@gmail.com}{ amreaves120@gmail.com}
$\mid$ \faLinkedinSquare
 \href{https://www.linkedin.com/in/alexander-reaves/}{ {alexander-reaves}} $\mid$ \faPhone \href{tel:+14802131323}{ +1 (480)  213 -- 1323}   } 


\begin{document}

%----------------------------------------------------------------------------------------
%	EDUCATION SECTION
%----------------------------------------------------------------------------------------


    Aerospace Engineering Ph.D. student with 5+ years of scientific research experience. Currently enrolled in the Future Propulsion and Power program at the Whittle Laboratory at the University of Cambridge. Strong technical expertise in turbomachinery design, engine design, and physics-based modeling.  U.S. citizen with 5 years international education.



%%%focus more on experience and add future research interests

\begin{rSection}{Education}
\begin{rSubsection}
    {University of Cambridge}{Cambridge, U.K.}{Ph.D. in Electric Aircraft Propulsor Design}{October 2022 {\textendash}  Present}
    \item Developed novel methodologies for the aerodynamic design of ducted fan and open-rotor propeller blades for electric aircraft.
    \item Retrofit an existing wind tunnel to be usable for testing of ducted fan and open-rotor propulsor designs. 
    \item Developed expertise in producing aerospace-quality components using large-format 3D printers.
    


%%% JVT - all of your bullet points for your research experience placements are amazing! Put some in here for your most recent education. THis is the first bit anyone will read and so needs to be AMAZING 
    
\end{rSubsection}
%{\bf \large  University of Cambridge} \hfill {Cambridge, U.K.}\\
%{\em Ph.D. in Future Propulsion and Power (Aerospace Engineering)} \hfill {October 2022 {\textendash}  Present}
\begin{rSubsection}{University of Cambridge}{Cambridge, U.K.}{ M.Res. in Future Propulsion and Power (Aerospace Engineering)}{October 2021 {\textendash}  August 2022}
\item  Advanced course in the aero-thermal engineering of propulsion and power devices, with an emphasis on the gas turbine (compressors, combustors, and turbines).
    \item Courses on experimental methods, writing C.F.D. solvers, machine-learning techniques, C.F.D. post-processing, and turbomachinery blade design.
\end{rSubsection}


%{\bf \large  University of Cambridge} \hfill {Cambridge, U.K.}\\
%{\em M.Res. in Future Propulsion and Power (Aerospace Engineering)} \hfill {October 2021 {\textendash}  August 2022}
% \\
%

{\bf \large  Yale-NUS College} \hfill { Singapore}\\
{\em B.S. with Honours in Physical Sciences (Physics)} \hfill { August 2017 {\textendash}  May 2021}

{\bf \large  University of Cambridge} \hfill {Cambridge, U.K.}\\
{\em Visiting Student (Mathematics)} \hfill {October 2019 {\textendash}  June 2020}


{\bf \large  Phoenix Country Day School} \hfill { Arizona, U.S.A.}\\
{\em High School Diploma} \hfill { August 2014 {\textendash}  June 2017}




\end{rSection}
%----------------------------------------------------------------------------------------
%	TECHNICAL STRENGTHS SECTION
%----------------------------------------------------------------------------------------

%%%% JVT - I don't think anyone really reads / cares about this section, either delete or move to the end of your CV %%%



%----------------------------------------------------------------------------------------
%	WORK EXPERIENCE SECTION
%----------------------------------------------------------------------------------------

\begin{rSection}{RESEARCH EXPERIENCE}
%\begin{rSubsection}{Yale-NUS Physical Sciences Department and University of Cambridge}{} {Senior Capstone Project} 
%{August 2020 {\textendash} Present}
%\item  Explored the feasibility of sand-powered energy generation
%\item 

%\end{rSubsection}
\begin{rSubsection}{Yale-NUS Physical Sciences Major}{Yale-NUS College \& University of Cambridge}{Capstone Student}{August 2020 {\textendash} May 2021}
\item Conducted a yearlong research project modeling the interactions between granular flows and hydropower turbines. 
\item Met weekly with supervisors from the University of Cambridge and Yale-NUS College and have written a draft paper.
%\item Wrote a thesis capturing the results of capstone research project. 

\end{rSubsection}


\begin{rSubsection}{NASA Ames Research Center}{National Aeronautics and Space Administration} {Summer Intern - Remote} 
{June  {\textendash} August 2020}
\item  Designed 3D models of components for the International Space Station using \texttt{Creo Parametric}.
\item 3D printed and tested over 20 iterations of a $\text{CO}_2$ sensor which could be manufactured in space.

\end{rSubsection}

\begin{rSubsection}{Yale-NUS Sciences Department}{Yale-NUS College}{Research Assistant for Professor Chelsea Sharon}{  June  {\textendash} August 2019}
\item Awarded funding of over SGD \$3,000 from the J.Y. Pillay Global-Asia Programme to research the feasibility of radio astronomy data collection in Singapore. 

\item Designed, constructed, and wrote code for a 1420 MHz horn radio telescope based off of the Bessie radio telescope design from Open Source Radio Telescopes. 

\end{rSubsection}

\begin{rSubsection}{Centre for Advanced 2D Materials}{National University of Singapore}{Research Assistant}{May 2018 {\textendash} August 2019}
\item Awarded full funding of over SGD \$3,000 from J.Y. Pillay Global-Asia Programme to research superconductivity in twisted bilayer graphene.
\item Coded and ran over 20 different simulations using \texttt{MATLAB} and \texttt{Python} to determine the electronic band properties of superconductive twisted bilayer graphene.
\item Presented relevant papers and research findings in group meetings and weekly journal clubs.
\item \href{https://doi.org/10.1016/j.ssc.2018.07.013} {Published results of research in \textit{Solid State Communications}. This work has received over 80 citations.}
\end{rSubsection}


%----------------------------------------------



%\begin{rSubsection}{ASU NEWSPACE}{Arizona State University}{Intern for Program Manager Scott Smas	}{	\normalfont{  May 2016 {\textendash} July 2016}}
%\item 	Researched commercial aerospace enterprises that would be interested in joining ASU NewSpace's MILO Institute initiative.
%\item Managed the transfer of ASU NewSpace's calendaring system from a pen-and-paper system to a cloud-accessible Google Calendar.

%\end{rSubsection}

\end{rSection}

\begin{rSection}{TEACHING EXPERIENCE}

\begin{rSubsection}
{Cambridge University Engineering Department}{University of Cambridge}{Student Supervisor (Teaching Assistant)}{October 2022 - March 2023}
\item Tutored 6 undergraduate students on the topics of 1D compressible flows, 2D compressible flows, numerical methods, and turbomachinery design. 
\item Supervised students in small groups (1-3 students) on problem sets and exam preparation questions. 
\item Assessed and recorded student progress and provided feedback to students and their academic advisors.
\end{rSubsection}

\begin{rSubsection}
{C.D.T. in Future Propulsion and Power}{University of Cambridge}{Demonstrator}{February 2023}
\item Provided guidance and feedback for 18 masters students on the use of \texttt{ParaView} for visualization of 3D CFD results of a compressor stator row. 
\item Mentored students on the use of advanced manufacturing techniques, such as 3D printing and rapid prototyping, to produce aerospace components.
\end{rSubsection}

\end{rSection}

\begin{rSection}{OTHER PROFESSIONAL EXPERIENCE }

%------------------------------------------------
\begin{rSubsection} {Open Ventilator System Initiative (OVSI)}{University of Cambridge}
 {Engineer / Engineering Coordinator }{March 2020 {\textendash} June 2020}
\item Collaborated on designing and prototyping 3 different versions of an affordable, hospital-quality, ventilator system that is manufacturable and maintainable in low and middle-income countries.
\item Managed information sharing and co-development between engineering groups in the United Kingdom, Kenya, Uganda, and Ethiopia.
\item Received President’s Special Award for Pandemic Service from Royal Academy of Engineering  for contributions to addressing the challenges of the COVID-19 pandemic. 
\end{rSubsection}

\begin{rSubsection}
{United Nations Office for Outer Space Affairs}{United Nations}{Online Volunteer}{	 December 2018 {\textendash} March 2019}

 \item Researched methodologies for wastewater recycling and water management and determined their feasibility to be applied toward UN Sustainable Development Goal 6: Sustainable Management of Water and Sanitation for All.
 
\item   Wrote scientific-communication articles which explain the potential applications of technologies developed for use in outer space for water management on Earth.
\end{rSubsection}

%\begin{rSubsection}{NOBLE MISSION FOR CHANGE INITIATIVE}{ Abuja, Nigeria}{United Nations Online Volunteer}{	\normalfont{  November 2018 {\textendash} January 2019}}
%\item  Created informative data visualizations which assist the Noble Mission for Change Initiative in its mission to improve access to quality education in Africa.
%\end{rSubsection}

%\begin{rSubsection}{SAGA COLLEGE, YALE-NUS COLLEGE }{Singapore}{Student Associate	}{  \normalfont{  2017 {\textendash} 2018}}
%\item Conceptualized and implemented community-building events ranging from monthly birthday parties to informal academic talks for a residential college community of over 800 undergraduates and faculty members.
%\item Oversaw the day-to-day logistics of office procedures, such as mail distribution and inventory management.
%\end{rSubsection}

\end{rSection}

\begin{rSection}{HONORS AND AWARDS}

\begin{rSubsection}
{President's Special Award for Pandemic Service}{2020}{}{}
\item Award given from the Royal Academy of Engineering to OVSI members for contributions to addressing the challenges of the COVID-19 pandemic. 
\end{rSubsection}

\newpage

\begin{rSubsection}
{JY Pillay Global-Asia Programme Research Award ($\times 2$) }{2018 \& 2019}{}{}
\item Received full funding of \$5000 SGD  to construct a radio telescope to test the suitability of  Singapore's RF environment for radio astronomy observations during summer 2019.
    \item Received full funding of \$5000 SGD to research superconductivity in twisted bilayer graphene during summer 2018.
\end{rSubsection}

%\textbf{JY Pillay Global-Asia Programme Research Award}\hfill 2018\\

% \\ \itab{Process Control (ongoing)} \tab{} \tab{Electrodynamics}
\end{rSection}

\begin{rSection}{TRAINING}

%%% JVT - this section is quite long, consider cutting down to 1 bullet point each (max 2).

\begin{rSubsection}
    {Vertical Flight Society}{Mesa, Arizona, U.S.A.}{Short Course on Electric VTOL Design Attendee}{January 2023}
    \item Attendee to the VFS's 2023 Short Course on Electric VTOL Design taught by Dr. James Wang. 
    \item Course topics included: initial design considerations and EVTOL architectures, battery and hybrid electric aircraft, electric motor design, weight and performance estimation, rotor design, stability \& control, testing, rotor and vehicle performance analysis, benchmarking and cost estimation, and certification and vertiport design.
\end{rSubsection}

\begin{rSubsection}
    {Rolls-Royce Plc.}{Derby, U.K.}{Student Trainee}{June 2022}
    \item Visits to the Rolls-Royce civil and defence sites in Bristol, Birmingham, and Derby. 
    \item Seminars on combustion systems, turbine design, EVTOL fan aerodynamics, engine control and health monitoring, and electronic/hydromechanical sub-systems.
\end{rSubsection}

\begin{rSubsection}
    {Dyson Ltd.}{Malmesbury, U.K.}{Student Trainee}{June 2022}
    \item One-day training focused on product design. Topics included small-scale radial turbomachinery, system integration, electrification, noise mitigation, and cost reduction.
\end{rSubsection}


\begin{rSubsection}
    {Siemens Energy AG.}{Lincoln, U.K.}{Student Trainee}{May 2022}
\item Three-week hands-on industrial training covering:  mechanical integrity and metallurgical assessment for turbomachinery blades; lean manufacturing; installation, commissioning, and servicing of land-based gas turbine engines; gearbox design; control systems design; and wind turbine technology.
\item Developed a safe and cost-effective method for transporting gas turbine engines within power plants. 
\end{rSubsection}


\begin{rSubsection}
    {University of Oxford}{Oxford, U.K.}{Visiting Student}{March - April 2022}
\item Three-week course on heat transfer and high-speed aerodynamics at the Oxford Thermofluids Institute. 

\item Designed and compared 4 different supersonic nozzle designs using 2D and 3D method of characteristics approaches. 

\item Performed liquid crystal thermography on a ribbed channel to determine wall heat transfer measurements and compared results with CFD results and thermocouple and total pressure traverse data.

\item Characterized the sensitivity to geometric perturbations of flow and heat transfer results using a single value decomposition (SVD) technique.

\end{rSubsection}




\begin{rSubsection}
    {University of Loughborough}{Loughborough, U.K.}{Visiting Student}{January 2022}
\item  Three-week course on combustion system aerodynamics at the National Centre for Combustion and Aerothermal Technology (NCCAT) at the University of Loughborough. 

\item Lectures on combustion system technologies, advanced research methods, and data analysis
techniques (e.g. conditional sampling/averaging, proper orthogonal decomposition, linear stochastic
estimation). 
\item Trained in experimental techniques for characterizing a fuel-spray nozzle at different technology readiness levels: aerodynamics of a fuel injector using particle image velocimetry (PIV); fuel injector spray characteristic using shadowgraphy.  
\end{rSubsection}
    
\end{rSection}


%	EXAMPLE SECTION
%----------------------------------------------------------------------------------------


%----------------------------------------------------------------------------------------
\begin{rSection}{PUBLICATIONS AND PRESENTATIONS} 
\begin{itemize}[leftmargin={0 pt}, label={}, topsep=0pt]

%%% JVT - Put this section right at end, just before references 

\item \href{https://doi.org/10.1016/j.ssc.2018.07.013}{\textbf{Singlet superconductivity enhanced by charge order in nested twisted bilayer graphene Fermi surfaces}  
Evan Laksono, Jia Ning Leaw, \textbf{Alexander Reaves}, Manraaj Singh, Xinyun Wang, Shaffique Adam, Xingyu Gu;
Solid State Communications,
Volume 282,
Pages 38-44,
October 2018\\
https://doi.org/10.1016/j.ssc.2018.07.013 }

\item \href{https://www.space4water.org/news/wastewater-recycling-iss-and-singapore}{\textbf{Wastewater recycling on the ISS and in Singapore} \\
\textbf{Alexander Reaves}; United Nations Office of Outer Space Affairs, Space4Water, February 2019\\ https://www.space4water.org/news/wastewater-recycling-iss-and-singapore}

\item \href{http://meetings.aps.org/Meeting/MAR19/Session/S14.10}{\textbf{Magnetotransport properties in twisted bilayer graphene at magic angle} \\
Evan Laksono,  \textbf{Alexander Reaves}, Manraaj Singh, Xingyu Gu, Jia Ning Leaw, Nimisha Raghuvanshi, Shaffique Adam; American Physical Society, Abstract: S14.00010, March 2019\\ http://meetings.aps.org/Meeting/MAR19/Session/S14.10}

\end{itemize}
\end{rSection}
%----------------------------------------------------------------------------------------




\begin{rSection}{Technical Skills}

\begin{tabular}{ @{} >{\bfseries}l @{\hspace{6ex}} l }
Computer Languages &  \texttt{C, C++, MATLAB, Python}\\
Software \& Tools &  \texttt{Creo, LabVIEW, Mathematica, Microsoft Office, ParaView}\\
 Operating Systems & \texttt{MacOS, Windows, Linux (Ubuntu, Rocky Linux)}\\
 Experimental Techniques & hotwire measurements, three- and five-hole probe calibration \\ &and data collection, particle image velocimetry (PIV), liquid\\
 & crystal thermography, shadography
\end{tabular}

\end{rSection}



%----------------------------------------------------------------------------------------


\begin{rSection}{References}


\textbf{James Taylor} $|$ 
Assistant Professor, Department of Engineering $|$
University of Cambridge\\
1 JJ Thomson Ave, Cambridge, United Kingdom, CB3 0DY\\
jvt24@cam.ac.uk $|$ 
+44 7950-852578

\textbf{Sam Grimshaw} $|$ 
Mitsubishi Heavy Industries Senior Research Fellow $|$
University of Cambridge\\
1 JJ Thomson Ave, Cambridge, United Kingdom, CB3 0DY\\
sdg33@cam.ac.uk $|$ 
+44 1223-337581

\textbf{Christopher Clark} $|$ 
Assistant Professor, Department of Engineering $|$
University of Cambridge\\
1 JJ Thomson Ave, Cambridge, United Kingdom, CB3 0DY\\
sdg33@cam.ac.uk $|$ 
+44 1223-337581


%\textbf{Chelsea Electra Sharon} $|$
%Assistant Professor, Physical Sciences $|$
%Yale-NUS College\\
%01-101, 10 College Avenue West, Singapore, 138609\\
%chelsea.sharon@yale-nus.edu.sg $|$
%+65 6601-7558


%\textbf{Zhuang Bilin} $|$
%Assistant Professor, Physical Sciences $|$
%Yale-NUS College\\
%01-101, 10 College Avenue West, Singapore, 138609\\
%zhuang.bilin@yale-nus.edu.sg $|$
%+65 6419-1275



\end{rSection}







%\let\thefootnote\relax\footnotetext{Last Updated: \today}

\end{document}
