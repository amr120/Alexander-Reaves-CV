%%%%%%%%%%%%%%%%%%%%%%%%%%%%%%%%%%%%%%%%%
% Medium Length Professional CV
% LaTeX Template
% Version 2.0 (8/5/13)
%
% This template has been downloaded from:
% http://www.LaTeXTemplates.com
%
% Original author:
% Trey Hunner (http://www.treyhunner.com/)
%
% Important note:
% This template requires the resume.cls file to be in the same directory as the
% .tex file. The resume.cls file provides the resume style used for structuring the
% document.
%
%%%%%%%%%%%%%%%%%%%%%%%%%%%%%%%%%%%%%%%%%

%----------------------------------------------------------------------------------------
%	PACKAGES AND OTHER DOCUMENT CONFIGURATIONS
%----------------------------------------------------------------------------------------

\documentclass{resume} % Use the custom resume.cls style
\usepackage{hyperref}
\usepackage[left=0.75in,top=0.6in,right=0.75in,bottom=0.6in]{geometry} % Document margins
\usepackage{enumitem}%http://ctan.org/pkg/enumitem
\usepackage{amsmath}
\usepackage{fontawesome}
\newcommand{\tab}[1]{\hspace{.2667\textwidth}\rlap{#1}}
\newcommand{\itab}[1]{\hspace{0em}\rlap{#1}}
% Your name
%\usepackage{color}
%\hypersetup{colorlinks=true, urlcolor = blue, linkcolor = black}


\name{Alexander Michael Reaves} 

% Your address
\address{\faEnvelope
\href{mailto:amr200@cam.ac.uk}{ amr200@cam.ac.uk}
$\mid$ \faLinkedinSquare
 \href{https://www.linkedin.com/in/alexander-reaves/}{ {alexander-reaves}} $\mid$ \faPhone \href{tel:+14802131323}{ +1 (480)  213 -- 1323}   } 


\begin{document}

%----------------------------------------------------------------------------------------
%	EDUCATION SECTION
%----------------------------------------------------------------------------------------

\begin{center}
    Aerospace Engineering Ph.D. student in the EPSRC CDT in Future Propulsion and Power at the University of Cambridge. 4+ years of scientific research experience. Strong technical expertise in turbomachinery, propulsor design, and physics-based modeling.  
\end{center}




%%%focus more on experience and add future research interests

\begin{rSection}{Education}

{\bf \large  University of Cambridge} \hfill {Cambridge, UK}\\
{\em Ph.D. in Future Propulsion and Power (Aerospace Engineering)} \hfill { October 2022 {\textendash}  Present }

{\bf \large  University of Cambridge} \hfill {Cambridge, UK}\\
{\em M.Res. in Future Propulsion and Power (Aerospace Engineering)} \hfill { October 2021 {\textendash}  August 2022 }


{\bf \large  Yale-NUS College} \hfill { Singapore}\\
{\em B.S. with Honours in Physical Sciences (Physics)} \hfill { August 2017 {\textendash}  May 2021 }



{\bf \large  Phoenix Country Day School} \hfill { Arizona, USA}\\
{\em High School Diploma} \hfill { August 2014 {\textendash}  June 2017}




\end{rSection}
%----------------------------------------------------------------------------------------


%----------------------------------------------------------------------------------------
%	WORK EXPERIENCE SECTION
%----------------------------------------------------------------------------------------

\begin{rSection}{RESEARCH EXPERIENCE}
%\begin{rSubsection}{Yale-NUS Physical Sciences Department and University of Cambridge}{} {Senior Capstone Project} 
%{August 2020 {\textendash} Present}
%\item  Explored the feasibility of sand-powered energy generation
%\item 

%\end{rSubsection}
\begin{rSubsection}{Yale-NUS Physical Sciences Major}{Yale-NUS College \& University of Cambridge}{Capstone Student}{August 2020 {\textendash} May 2021}
\item Conducted a yearlong research project modeling the interactions between granular flows and hydropower turbines. 
\item Met weekly with supervisors from the University of Cambridge and Yale-NUS College to present relevant findings. 
%\item Wrote a thesis capturing the results of capstone research project. 

\end{rSubsection}


\begin{rSubsection}{NASA Ames Research Center}{National Aeronautics and Space Administration} {Summer Intern} 
{June  {\textendash} August 2020}
\item  Designed 3D models of components for the International Space Station using \texttt{Creo Parametric}.
\item 3D printed and tested over 20 iterations of a $\text{CO}_2$ sensor which could be manufactured in space.

\end{rSubsection}

\begin{rSubsection}{Yale-NUS Sciences Department}{Yale-NUS College}{Research Assistant for Professor Chelsea Sharon}{  June  {\textendash} August 2019}
\item Awarded funding of over SGD \$3,000 from the J.Y. Pillay Global-Asia Programme to research the feasibility of radio astronomy data collection in Singapore. 

\item Designed, constructed, and wrote code for a 1420 MHz horn radio telescope based off of the Bessie radio telescope design from Open Source Radio Telescopes. 

\end{rSubsection}

\begin{rSubsection}{Centre for Advanced 2D Materials}{National University of Singapore}{Research Assistant}{May 2018 {\textendash} August 2019}
\item Awarded full funding of over SGD \$3,000 from J.Y. Pillay Global-Asia Programme to research superconductivity in twisted bilayer graphene.
\item Coded and ran over 20 different simulations using \texttt{MATLAB} and \texttt{Python} to determine the electronic band properties of superconductive twisted bilayer graphene.
\item Presented relevant papers and research findings in group meetings and weekly journal clubs.
\item \href{https://doi.org/10.1016/j.ssc.2018.07.013} {Published results of research in \textit{Solid State Communications}. This work has received over 60 citations.}
\end{rSubsection}

\newpage
%----------------------------------------------



%\begin{rSubsection}{ASU NEWSPACE}{Arizona State University}{Intern for Program Manager Scott Smas	}{	\normalfont{  May 2016 {\textendash} July 2016}}
%\item 	Researched commercial aerospace enterprises that would be interested in joining ASU NewSpace's MILO Institute initiative.
%\item Managed the transfer of ASU NewSpace's calendaring system from a pen-and-paper system to a cloud-accessible Google Calendar.

%\end{rSubsection}






\end{rSection}



\begin{rSection}{OTHER PROFESSIONAL EXPERIENCE }

%------------------------------------------------
\begin{rSubsection} {Open Ventilator System Initiative (OVSI)}{University of Cambridge}
 {Engineer / Engineering Coordinator }{March 2020 {\textendash} June 2020}
\item Collaborated on designing and prototyping 3 different versions of an affordable, hospital-quality, ventilator system that is manufacturable and maintainable in low and middle-income countries.
\item Managed information sharing and co-development between engineering groups in the United Kingdom, Kenya, Uganda, and Ethiopia.
\item Received President’s Special Award for Pandemic Service from Royal Academy of Engineering  for contributions to addressing the challenges of the COVID-19 pandemic. 
\end{rSubsection}

\begin{rSubsection}
{United Nations Office for Outer Space Affairs}{United Nations}{Online Volunteer}{	 December 2018 {\textendash} March 2019}

 \item Researched the applicability og methodologies for wastewater recycling and water management to achieve UN Sustainable Development Goal 6: Sustainable Management of Water and Sanitation for All.
 
\item   Wrote scientific-communication articles which explain the potential applications of technologies designed for use in outer space for water management on Earth.
\end{rSubsection}

%\begin{rSubsection}{NOBLE MISSION FOR CHANGE INITIATIVE}{ Abuja, Nigeria}{United Nations Online Volunteer}{	\normalfont{  November 2018 {\textendash} January 2019}}
%\item  Created informative data visualizations which assist the Noble Mission for Change Initiative in its mission to improve access to quality education in Africa.
%\end{rSubsection}

%\begin{rSubsection}{SAGA COLLEGE, YALE-NUS COLLEGE }{Singapore}{Student Associate	}{  \normalfont{  2017 {\textendash} 2018}}
%\item Conceptualized and implemented community-building events ranging from monthly birthday parties to informal academic talks for a residential college community of over 800 undergraduates and faculty members.
%\item Oversaw the day-to-day logistics of office procedures, such as mail distribution and inventory management.
%\end{rSubsection}

\end{rSection}


%	EXAMPLE SECTION
%----------------------------------------------------------------------------------------
\begin{rSection}{PUBLICATIONS AND PRESENTATIONS} 
\begin{itemize}[leftmargin={0 pt}, label={}, topsep=0pt]

\item \href{https://doi.org/10.1016/j.ssc.2018.07.013}{\textbf{Singlet superconductivity enhanced by charge order in nested twisted bilayer graphene Fermi surfaces}  
Evan Laksono, Jia Ning Leaw, \textbf{Alexander Reaves}, Manraaj Singh, Xinyun Wang, Shaffique Adam, Xingyu Gu;
Solid State Communications,
Volume 282,
Pages 38-44,
October 2018\\
https://doi.org/10.1016/j.ssc.2018.07.013 }

\item \href{https://www.space4water.org/news/wastewater-recycling-iss-and-singapore}{\textbf{Wastewater recycling on the ISS and in Singapore} \\
\textbf{Alexander Reaves}; United Nations Office of Outer Space Affairs, Space4Water, February 2019\\ https://www.space4water.org/news/wastewater-recycling-iss-and-singapore}

\item \href{http://meetings.aps.org/Meeting/MAR19/Session/S14.10}{\textbf{Magnetotransport properties in twisted bilayer graphene at magic angle} \\
Evan Laksono,  \textbf{Alexander Reaves}, Manraaj Singh, Xingyu Gu, Jia Ning Leaw, Nimisha Raghuvanshi, Shaffique Adam; American Physical Society, Abstract: S14.00010, March 2019\\ http://meetings.aps.org/Meeting/MAR19/Session/S14.10}

\end{itemize}
\end{rSection}
%----------------------------------------------------------------------------------------
\begin{rSection}{HONORS AND AWARDS}

\begin{rSubsection}
{President's Special Award for Pandemic Service}{2020}{}{}
\item Award given to OVSI from the Royal Academy of Engineering for contributions to addressing the challenges of the COVID-19 pandemic. 
\end{rSubsection}


\begin{rSubsection}
{JY Pillay Global-Asia Programme Research Award}{2018 \& 2019}{}{}
\item Received full funding  to construct a radio telescope to test the suitability of  Singapore's RF environment for radio astronomy observations during summer 2019.
    \item Received full funding to research superconductivity in twisted bilayer graphene during summer 2018.
\end{rSubsection}


%\textbf{JY Pillay Global-Asia Programme Research Award}\hfill 2018\\


% \\ \itab{Process Control (ongoing)} \tab{} \tab{Electrodynamics}

\end{rSection}



%	TECHNICAL STRENGTHS SECTION
%----------------------------------------------------------------------------------------

\begin{rSection}{Technical Strengths}

\begin{tabular}{ @{} >{\bfseries}l @{\hspace{6ex}} l }
Computer Languages &  \texttt{C, C++, MATLAB, Python}\\
Software \& Tools &  \texttt{Creo, LabVIEW, Mathematica, Microsoft Office}
%\\ Operating Systems & MacOS, Windows, Ubuntu
\end{tabular}

\end{rSection}
%----------------------------------------------------------------------------------------









\let\thefootnote\relax\footnotetext{Last Updated: \today}

\end{document}
